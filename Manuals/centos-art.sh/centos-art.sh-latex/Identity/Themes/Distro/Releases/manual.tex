%
% Describe The CentOS Distribution Release Schema.
%
\documentclass{article}

\usepackage{hyperref}

\title{The CentOS Release Brand}
\author{Alain Reguera Delgado}

\begin{document}

\maketitle

\begin{abstract}
This is abstract.
\end{abstract}

\tableofcontents

\section{Introduction}

The upstream vendor has released 3 versions of Enterprise Linux that
CentOS Project rebuilds the freely available SRPMS for\footnote{
\url{http://wiki.centos.org/About}}. So, the major CentOS releases are
CentOS 3, CentOS 4 and CentOS 5. The upstream vendor releases security
updates as required by circumstances. CentOS Project releases rebuilds
of security updates as soon as possible. Usually within 24 hours (our
stated goal is with 72 hours, but we are usually much faster).

The upstream vendor also releases numbered update sets for Version 3,
Version 4 and Version 5 of their product (i.e. EL 3 update 9, EL 4
update 6 and EL 5 update 1) 2 to 4 times per year. There are new ISOs
from the upstream vendor provided for these update sets. Update sets
will be completed as soon as possible after the vendor releases their
version\ldots generally within 2 weeks. CentOS Project follows these
conventions as well, so CentOS 3.9 correlates with EL 3 update 9 and
CentOS 4.6 correlates with EL 4 update 6, CentOS 5.1 correlates to EL
5 update 1, etc.

One thing some people have problems understanding is that if you have
any CentOS-3 product and update it, you will be updated to the latest
CentOS-3.x version. The same is true for CentOS-4 and CentOS 5. If you
update any CentOS-4 product, you will be updated to the latest
CentOS-4.x version, or to the latest CentOS 5.x version if you are
updating a CentOS 5 system.  This is exactly the same behavior as the
upstream product. Let's assume that the latest EL4 product is update
6. If you install the upstream original EL4 CDs (the ones before any
update set) and upgrade via their up2date, you will have latest update
set installed (EL4 update 6 in our example). 

Since all updates within a major release (CentOS 3, CentOS 4, CentOS
5) always upgrade to the latest version when updates are performed
(thus mimicking upstream behavior), only the latest version is
maintained in each main tree on the CentOS
Mirrors\footnote{\url{http://mirrors.centos.org/}}.

There is a CentOS Vault\footnote{\url{http://vault.centos.org/}}
containing old CentOS trees. This vault is a picture of the older tree
when it was removed from the main tree, and does not receive updates.
It should only be used for reference. 

\section{Workplace}

\begin{itemize}
\item SVN:trunk/Identity/Logos/svg/type/2c-tmr.svg
\item SVN:trunk/Identity/Logos/svg/type/build/tmr3.svg
\item SVN:trunk/Identity/Logos/svg/type/build/tmr4.svg
\item SVN:trunk/Identity/Logos/svg/type/build/tmr5.svg
\end{itemize}

\section{Design}

It is very important that people differentiate which is the major
release of CentOS Distribution they are using. To achive this, we use
a special brand called \textit{The Release Brand} of CentOS
Distribution.

There is one Release Brand for each Major Release of CentOS
Distribution.  The Release Brand of CentOS Distribution is placed on
images controlling the CentOS Distribution Visual Style.

The Release Brand of CentOS Distribution is built using two
components: 1. The CentOS Trademark, 2. The Major Release Number of
CentOS Distribution.

The height of the Release Number is twice the CentOS Trademark height
and it is placed on the right side of CentOS Trademark, both bottom
aligned.

Sometimes The CentOS Message can be added as third component to The
Release Brand.  In these cases The CentOS Message remains on English
language, it is not translated.  Because of this, The Release Brand
that includes The CentOS Message should be avoided or used in places
where there is no posibility for the user to select a different
language but English.  Examples of these kind of images are Anaconda
Prompt and GRUB.

\section{Rendering}

\end{document}
