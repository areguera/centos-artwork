\documentclass{article}
\usepackage{longtable}
\usepackage[pdftex]{graphicx}
\usepackage{hyperref}
\hypersetup{pdfauthor={CentOS Documentation SIG},%
            pdftitle={Grand Unified Bootloader (GRUB)},%
            pdfsubject={CentOS Corporate Visual Identity}%
            }

\title{Grand Unified Bootloader (GRUB)}
\author{CentOS Documentation SIG}

\begin{document}

\maketitle

\begin{abstract} 
This article describes GRUB Visual Style for CentOS Distribution.
This screen is where the selection of which kernel to run and other
boot-time options, are made. It is seen every time the computer boots.

Copyright \copyright\ 2010\ The CentOS Project. Permission is
granted to copy, distribute and/or modify this document under the
terms of the GNU Free Documentation License, Version 1.2 or any later
version published by the Free Software Foundation; with no Invariant
Sections, no Front-Cover Texts, and no Back-Cover Texts. A copy of the
license is included in the section entitled ``GNU Free Documentation
License''.  
\end{abstract}

\tableofcontents

\section{Workplace}

\begin{itemize}
\item SVN:trunk/Identity/Themes/\$THEME/Distro/BootUp/GRUB/
\item SVN:trunk/Translations/Identity/Themes/Distro/BootUp/GRUB/
\item SVN:trunk/Scripts/Identity/Themes/Distro/BootUp/GRUB/
\end{itemize}

\section{Theme}

\begin{longtable}{rl}
\hline
\multicolumn{2}{l}{\ }\\
\textbf{Target}: & /usr/share/gdm/themes/Modern/background.png\\
\textbf{Package}: & \textbf{Unknown!}\\
\textbf{Description}: & PNG image data, 1024 x 768, 8-bit/color RGBA, non-interlaced\\
\multicolumn{2}{l}{\ }\\
\textbf{Target}: & /usr/share/gdm/themes/Modern/centos-release.png\\
\textbf{Package}: & \textbf{Unknown!}\\
\textbf{Description}: & PNG image data, 181 x 48, 8-bit/color RGBA, non-interlaced\\
\multicolumn{2}{l}{\ }\\
\textbf{Target}: & /usr/share/gdm/themes/Modern/centos-symbol.png\\
\textbf{Package}: & \textbf{Unknown!}\\
\textbf{Description}: & PNG image data, 48 x 48, 8-bit/color RGBA, non-interlaced\\
\multicolumn{2}{l}{\ }\\
\textbf{Target}: & /usr/share/gdm/themes/Modern/GdmGreeterTheme.desktop\\
\textbf{Package}: & \textbf{Unknown!}\\
\textbf{Description}: & UTF-8 Unicode English text\\
\multicolumn{2}{l}{\ }\\
\textbf{Target}: & /usr/share/gdm/themes/Modern/icon-language.png\\
\textbf{Package}: & \textbf{Unknown!}\\
\textbf{Description}: & PNG image data, 32 x 32, 8-bit/color RGBA, non-interlaced\\
\multicolumn{2}{l}{\ }\\
\textbf{Target}: & /usr/share/gdm/themes/Modern/icon-reboot.png\\
\textbf{Package}: & \textbf{Unknown!}\\
\textbf{Description}: & PNG image data, 32 x 32, 8-bit/color RGBA, non-interlaced\\
\multicolumn{2}{l}{\ }\\
\textbf{Target}: & /usr/share/gdm/themes/Modern/icon-session.png\\
\textbf{Package}: & \textbf{Unknown!}\\
\textbf{Description}: & PNG image data, 32 x 32, 8-bit/color RGBA, non-interlaced\\
\multicolumn{2}{l}{\ }\\
\textbf{Target}: & /usr/share/gdm/themes/Modern/icon-shutdown.png\\
\textbf{Package}: & \textbf{Unknown!}\\
\textbf{Description}: & PNG image data, 32 x 32, 8-bit/color RGBA, non-interlaced\\
\multicolumn{2}{l}{\ }\\
\textbf{Target}: & /usr/share/gdm/themes/Modern/Modern.xml\\
\textbf{Package}: & \textbf{Unknown!}\\
\textbf{Description}: & XML 1.0 document text\\
\multicolumn{2}{l}{\ }\\
\textbf{Target}: & /usr/share/gdm/themes/Modern/screenshot.png\\
\textbf{Package}: & \textbf{Unknown!}\\
\textbf{Description}: & PNG image data, 200 x 150, 8-bit/color RGBA, non-interlaced\\
\multicolumn{2}{l}{\ }\\
\hline
\end{longtable}


\section{Design}

Initially, \emph{splash.xpm.gz} is a PNG image (splash.png) which is
converted to xpm.gz. \emph{splash.png} image is rendered for each
major release of CentOS distribution.  Each image is based in the same
Artistic Motif and has the following components:

\begin{enumerate}
\item The CentOS Release Brand.
\item The CentOS Default Artistic Motif.
\end{enumerate}

Image rendering is done using the rendering script (\emph{render.sh})
available in the workplace of this section.  This script creates
the appropriate PNG images under \emph{img/\$VERSION/}
directory.

After image rendering, each \emph{img/\$VERSION/splash.png} image
should be indexed to 14 colors. This can be done using an image
manipulation tool like GIMP, or ImageMagick.  This color reduction
could bring some noise to your design. If that is the case, you need
to retouch your design in a 14 colors basis.

The final step is to convert the 14 colors indexed \emph{splash.png}
image into \emph{splash.xpm.gz}. To do so, use the command
\emph{convert2xpm.sh} provided in the workplace. This command
explores the \emph{img/\$VERSION/} directories and
converts\footnote{\emph{convert splash.png splash.xpm \&\& gzip
splash.xpm}} each \emph{splash.png} image indexed to 14 colors to its
\emph{.xpm.gz} equivalent. The converted images are saved under
\emph{xpm/\$VERSION/} directories. 

\section{Configuration}
\section{Rendering}
\section{Testing}
\section{Issues}

The following issues were seen on a video card \emph{Trident
Microsysmtes CyberBlade/i1 (cyblafb)}:

\begin{description}

\item[Different colors]: As more different colors you have on your
design, more are the possibilities of increasing the amount of noise
in your design after indexing to 14 colors. For example, if you
include the actual CentOS symbol in this image, it ocupies 3 colors
(for the orange, green, violet) which are completely different and
non-reusable in the blue toned background image.

\item [CentOS Symbol]: If the CentOS symbol is included in
this image, colors used in the symbol after indexing the image
are not the defaults colors defined as CentOS Symbol Colors. 

To workaround this, in first place, I used a variant of CentOS symbol
without background colors, just the white borders. Later, I desided to
remove it completely because that symbol could confuse people about
which is the CentOS default symbol (see ``The CentOS Brand'' manual).
Finally, I ended up using just the plain word CentOS to brand the
GRUB. 

\end{description}

% License section
\input{../../../../../Licenses/GFDL.tex}
\end{document}
