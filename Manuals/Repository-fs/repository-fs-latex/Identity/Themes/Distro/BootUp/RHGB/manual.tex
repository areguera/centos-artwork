\documentclass{article}
\usepackage{longtable}
\usepackage[pdftex]{graphicx}
\usepackage{hyperref}
\hypersetup{pdfauthor={CentOS Documentation SIG},%
            pdftitle={Graphic Boot (RHGB))},%
            pdfsubject={CentOS Corporate Visual Identity}%
            }

\title{Graphic Boot (RHGB)}
\author{CentOS Documentation SIG}

\begin{document}

\maketitle

\begin{abstract} 

This article describes the RHGB Visual Style for CentOS Distribution.
This screen is shown as the machine starts up. Users can toggle
between graphic ``Hide Detail'' mode text ``Show Detail'' mode.

Copyright \copyright\ 2010\ The CentOS Project. Permission is
granted to copy, distribute and/or modify this document under the
terms of the GNU Free Documentation License, Version 1.2 or any later
version published by the Free Software Foundation; with no Invariant
Sections, no Front-Cover Texts, and no Back-Cover Texts. A copy of the
license is included in the section entitled ``GNU Free Documentation
License''.  
\end{abstract}

\tableofcontents

\section{Workplace}

\begin{itemize}
\item SVN:trunk/Identity/Themes/\$THEME/Distro/BootUp/RHGB/
\item SVN:trunk/Translations/Identity/Themes/Distro/BootUp/RHGB/
\item SVN:trunk/Scripts/Identity/Themes/Distro/BootUp/RHGB/
\end{itemize}

\section{Theme}

\begin{longtable}{rl}
\hline
\multicolumn{2}{l}{\ }\\
\textbf{Target}: & /usr/share/gdm/themes/Modern/background.png\\
\textbf{Package}: & \textbf{Unknown!}\\
\textbf{Description}: & PNG image data, 1024 x 768, 8-bit/color RGBA, non-interlaced\\
\multicolumn{2}{l}{\ }\\
\textbf{Target}: & /usr/share/gdm/themes/Modern/centos-release.png\\
\textbf{Package}: & \textbf{Unknown!}\\
\textbf{Description}: & PNG image data, 181 x 48, 8-bit/color RGBA, non-interlaced\\
\multicolumn{2}{l}{\ }\\
\textbf{Target}: & /usr/share/gdm/themes/Modern/centos-symbol.png\\
\textbf{Package}: & \textbf{Unknown!}\\
\textbf{Description}: & PNG image data, 48 x 48, 8-bit/color RGBA, non-interlaced\\
\multicolumn{2}{l}{\ }\\
\textbf{Target}: & /usr/share/gdm/themes/Modern/GdmGreeterTheme.desktop\\
\textbf{Package}: & \textbf{Unknown!}\\
\textbf{Description}: & UTF-8 Unicode English text\\
\multicolumn{2}{l}{\ }\\
\textbf{Target}: & /usr/share/gdm/themes/Modern/icon-language.png\\
\textbf{Package}: & \textbf{Unknown!}\\
\textbf{Description}: & PNG image data, 32 x 32, 8-bit/color RGBA, non-interlaced\\
\multicolumn{2}{l}{\ }\\
\textbf{Target}: & /usr/share/gdm/themes/Modern/icon-reboot.png\\
\textbf{Package}: & \textbf{Unknown!}\\
\textbf{Description}: & PNG image data, 32 x 32, 8-bit/color RGBA, non-interlaced\\
\multicolumn{2}{l}{\ }\\
\textbf{Target}: & /usr/share/gdm/themes/Modern/icon-session.png\\
\textbf{Package}: & \textbf{Unknown!}\\
\textbf{Description}: & PNG image data, 32 x 32, 8-bit/color RGBA, non-interlaced\\
\multicolumn{2}{l}{\ }\\
\textbf{Target}: & /usr/share/gdm/themes/Modern/icon-shutdown.png\\
\textbf{Package}: & \textbf{Unknown!}\\
\textbf{Description}: & PNG image data, 32 x 32, 8-bit/color RGBA, non-interlaced\\
\multicolumn{2}{l}{\ }\\
\textbf{Target}: & /usr/share/gdm/themes/Modern/Modern.xml\\
\textbf{Package}: & \textbf{Unknown!}\\
\textbf{Description}: & XML 1.0 document text\\
\multicolumn{2}{l}{\ }\\
\textbf{Target}: & /usr/share/gdm/themes/Modern/screenshot.png\\
\textbf{Package}: & \textbf{Unknown!}\\
\textbf{Description}: & PNG image data, 200 x 150, 8-bit/color RGBA, non-interlaced\\
\multicolumn{2}{l}{\ }\\
\hline
\end{longtable}


The system-logo.png image is rendered for each major release of
CentOS. This task is done using the script render.sh available in the
workplace.  This script creates the appropriate PNG images under
img/\$VERSION/ directory. 

\section{Design}
\section{Configuration}
\section{Rendering}
\section{Testing}
\section{Issues}

% License section
\input{../../../../../Licenses/GFDL.tex}

\end{document}
