%
% Describe The CentOS Motif.
%
\documentclass{article}
\usepackage{hyperref}

\title{The CentOS Motif}
\author{Alain Reguera Delgado}

\begin{document}

\maketitle

\begin{abstract}
This manual describes the workplace, design, and rendering of CentOS
Artistic Motifs, both default and alternatives.
\end{abstract}

\tableofcontents

\section{Introduction}

The CentOS Artistic Motif is an image used to enforce the CentOS
Project Visual Identity. The Artistic Motif is a pattern design used
to define the CentOS Visual Style.

Due to our Monolithic Visual Structure, the CentOS Project's Visual
Identity is attached to one unique Visual Style, that is the CentOS
Default Visual Style. CentOS Default Visual Style is based on one
unique CentOS Artistic Motif, that is the CentOS Default Artistic
Motif.

Changing the CentOS Default Visual Style is not very convenient
because that affects the ``recognition'' of CentOS Project.
Nevertheless, we want to see what do you have.  Specially if your work
is an improvement to the base idea of CentOS Default Visual Style
(\emph{\textbf{Modern}, squares and circles flowing up.}).

Additionally to the CentOS Default Artistic Motif, there are CentOS
Alternative Motifs.  CentOS Alternative Motifs may or may not be
related with the current CentOS Default Artistic Motif.  CentOS
Alternative Motfis are an space for new art creation, for designing
new and completely exiting artistic ideas.  This place doesn't pretend
to replace sites like devianart.org, but to collect Artistic Motifs
focused on The CentOS Project and what it is.

If you are not happy with the actual CentOS Default Artistic Motif,
you can look inside CentOS Alternative Motifs and if someone is
interesting enough you can download it from the CentOS Artwork
Repository and test it. If it turns popular enough it has posibilities
of become the CentOS Default Artistic Motif and by extension the
CentOS Default Visual Style.

If you are not happy with CentOS Alternative Motifs either, then go an
design your own CentOS Alternative Artistic Moif and propose it in
\href{mailto:centos-devel@centos.org}{centos-devel@centos.org}.

CentOS Default Artistic Motif and CentOS Alternative Motifs are
maintain by CentOS Community People. Generally, one person proposes
the first idea, later others join the effort to make that idea better.
The first person who proposes the idea is known as the Motif Author
and is she/he who owns the copyright of that work. People joinning the
effort are known as Motif Contributors. 

The CentOS Project is using the Creative Common Share-Alike
License\footnote{http://creativecommons.org/licenses/by-sa/3.0/} in
both CentOS Default Artistic Motif and CentOS Alternative Motifs. This
is, in order to brand an Artistic Motif as CentOS Motif, her/his
author should release her/his work under the previously mentioned
license. 

Only Artistic Motifs branded as CentOS Motif, both Default and
Alternatives, are hosted on CentOS Artwork Repository.

\section{Workplace}

\begin{itemize}
\item SVN:trunk/Identity/Themes/\$THEME/Motif/
\item SVN:trunk/Manuals/Identity/Themes/Motif/
\end{itemize}

\section{Design}

\subsection{The CentOS Motif Brand}

\subsection{Recommendations}

When designing Motifs for CentOS, consider the following
recommendations:

\begin{itemize}

\item Give a unique (case-sensitive) name to your Motif. This name is
used as value wherever \$THEME variable is. Optionally, you can add a
description about inspiration and concepts behind your work.

\item Use the location SVN:trunk/Identity/Themes/\$THEME/Motif/ to
store your work. If it doesn't exist create it. Note that this require
you to have previous commit access in CentOS Artwork Repository.

\item Use the CentOS Default Artistic Motif's Palette as base to your
work. CentOS Palette is available at
SVN:trunk/Identity/Themes/Modern/Palettes/Default.gpl.

\item Make your work completely vectorial. Do not add raster images
inside it.

\item Feel free to make your art enterprise-level and beautiful.

\item Add the following information on your artwork (both in a visible
design area, and inside inkscape document metadata section wherever it
be possible):

\begin{itemize}
\item The CentOS Motif Brand.
\item The name of your artistic motif.
\item The copyright sentence: \texttt{Copyright (C) YEAR YOURNAME}
\item The license under which the work is released. 
\end{itemize}
\end{itemize}

\section{Rendering}

\end{document}
