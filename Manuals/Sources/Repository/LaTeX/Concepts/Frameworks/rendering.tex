% Part   : Concepts
% Chapter: Frameworks
% ------------------------------------------------------------
% $Id: rendering.tex 6023 2010-06-27 10:09:48Z al $
% ------------------------------------------------------------

\section{Rendering}
\hypertarget{sec:Concepts:Frameworks:Rendering}{}

Rendering is the process by which you produce translated content based
on design templates and translation files. Inside CentOS Artwork
Repository you can render images and texts.

\subsection{Image Rendering}
\hypertarget{sec:Concepts:Frameworks:Rendering:Image}{}

Image files are rendered using the \texttt{render.sh} identity script.
The \texttt{render.sh} identity script is available in the framework
containing the image files you want to produce.  To execute the
\texttt{render.sh} identity script, you need to be inside framework's
directory and use the following syntax:

\begin{quote}
\texttt{./render.sh 'REGEX'}
\end{quote}

The REGEX argument is optional.  It is used to reduce the amount of
files you want to render.  It is a posix-egrep regular expression
pattern, applied against the translation path.

\subsection{Text Rendering}

Text files are rendered using the \texttt{render.sh} identity script.
The \texttt{render.sh} identity script is available in the framework
containing the text files you want to produce.  To execute the
\texttt{render.sh} identity script, you need to be inside framework's
directory and use the following syntax:

\begin{quote}
\texttt{./render.sh 'REGEX'}
\end{quote}

The REGEX argument is optional.  It is used to reduce the amount of
files you want to render.  It is a posix-egrep regular expression
pattern, applied against the translation path.
