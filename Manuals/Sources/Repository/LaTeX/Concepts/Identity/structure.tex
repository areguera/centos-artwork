% ------------------------------------------------------------
% $Id: structure.tex 6024 2010-06-28 04:28:27Z al $
% ------------------------------------------------------------
    \section{Visual Structure}
\hypertarget{sec:Concepts:Identity:Structure}{}
      \label{sec:Concepts:Identity:Structure}

The CentOS Project settles down its corporate visual identity on a
``monolithic corporate visual identity structure''. In this structure
The CentOS Project uses one unique name
(``\hyperlink{sec:Concepts:Identity:Brands}{The CentOS Brand}'') and
one unique visual style
(``\hyperlink{sec:Concepts:Identity:Themes:Default}{The CentOS Default
Theme}'') in all its manifestations. 

The CentOS Project organizes its visual manifestation in four top
level structures: The CentOS Distribution, The CentOS Web, The CentOS
Promotion, and The CentOS Behaiviour.  The CentOS Distribution , The
CentOS Web, and The CentOS Promotion use one unique name and one
unique visual style in all its manifestations.

% ------------------------------------------------------------
 \subsection{Distribution Visual Structure}
\hypertarget{sec:Concepts:Identity:Structure:Distribution}{}
      \label{sec:Concepts:Identity:Structure:Distribution}

It applies to all major releases of CentOS distribution.

Sometimes, specific visual manifestations are formed by common
components which have internal differences. That is the case of CentOS
Distribution visual manifestation.  

Since a visual style point of view, CentOS Distributions share common
artwork components like Anaconda ---to cover the CentOS distribution
installation---, BootUp ---to cover the CentOS distribution start
up---, and Backgrounds ---to cover the CentOS distribution desktop---.
Now, since a technical point of view, those common components are made
of software improved constantly. 

The software constant improvement is reflected on a numbered release
schema, described in ``\hyperlink{sec:Concepts:CentOS:Release}{The
CentOS Release Schema}'' (\autoref{sec:Concepts:CentOS:Release}). The CentOS release schema is a tool to
provide exact information, specific to one release at any given time. 

People can use this release schema to know the software details that
they are using on their computers, report bugs, fixes, suggestions, or
simply any kind of usefull information; in the same exact basis.

Remarking the CentOS release schema inside each major release of
CentOS Distribution ---or similar visual manifestation--- takes high
attention in the sake of The CentOS Project corporate visual identity.
For archiving that purpose, graphic designers use ``The CentOS Release
Brand'' in all artwork components controlling the visual style of
CentOS Distribution ---or similar--- visual manifestation.

Artwork components controlling the visual style of CentOS Distribution
visual manifestation are described in
``\hyperlink{par:Distribution}{Distribution}''
(\autoref{par:Distribution}).

% ------------------------------------------------------------
 \subsection{Web Visual Structure}
\hypertarget{sec:Concepts:Identity:Structure:Web}{}
      \label{sec:Concepts:Identity:Structure:Web}

It applies to all web applications CentOS uses to handle its needs
(Ex. Portals, Wikis, Forums, Blogs, Bug Tracker). Anything involving
HTML standards should be consider here.

% ------------------------------------------------------------
 \subsection{Promotion Visual Structure}
\hypertarget{sec:Concepts:Identity:Structure:Promotion}{}
      \label{sec:Concepts:Identity:Structure:Promotion}

It applies to all tangible and non tangible items CentOS uses to
promote its existence. Clothes, posters, installation media,
stationery, release countdown images, banners, stickers, are all
examples of promotion designs.

% ------------------------------------------------------------
 \subsection{Behaviour Visual Structure}
\hypertarget{sec:Concepts:Identity:Structure:Behaviour}{}
      \label{sec:Concepts:Identity:Structure:Behaviour}

It applies to CentOS community's social behavior. To what we do and
how we do it. 

