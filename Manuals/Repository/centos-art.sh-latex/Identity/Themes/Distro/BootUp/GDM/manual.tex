\documentclass{article}
\usepackage{longtable}
\usepackage[pdftex]{graphicx}
\usepackage{hyperref}
\hypersetup{pdfauthor={CentOS Documentation SIG},%
            pdftitle={GNOME Display Manager (GDM)},%
            pdfsubject={CentOS Corporate Visual Identity}%
            }

\title{GNOME Display Manager (GDM)}
\author{CentOS Documentation SIG}

\begin{document}

\maketitle

\begin{abstract} 
This article describes the GDM Visual Style for CentOS Distribution.
GDM is the first choice defined as CentOS Display Manager (DM). If
users leave their computers on all the time and don't share their
system, they won't see this as often as users who share a desktop
system with other users on the system or laptop users who reboot and
login more frequently.

Copyright \copyright\ 2010\ The CentOS Project. Permission is
granted to copy, distribute and/or modify this document under the
terms of the GNU Free Documentation License, Version 1.2 or any later
version published by the Free Software Foundation; with no Invariant
Sections, no Front-Cover Texts, and no Back-Cover Texts. A copy of the
license is included in the section entitled ``GNU Free Documentation
License''.  
\end{abstract}

\tableofcontents

\section{Workplace}

\begin{itemize}
\item SVN:trunk/Identity/Themes/\$THEME/Distro/BootUp/GDM/
\item SVN:trunk/Translations/Identity/Themes/Distro/BootUp/GDM/
\item SVN:trunk/Scripts/Identity/Themes/Distro/BootUp/GDM/
\end{itemize}

\section{Theme}

\begin{longtable}{rl}
\hline
\multicolumn{2}{l}{\ }\\
\textbf{Target}: & /usr/share/gdm/themes/Modern/background.png\\
\textbf{Package}: & \textbf{Unknown!}\\
\textbf{Description}: & PNG image data, 1024 x 768, 8-bit/color RGBA, non-interlaced\\
\multicolumn{2}{l}{\ }\\
\textbf{Target}: & /usr/share/gdm/themes/Modern/centos-release.png\\
\textbf{Package}: & \textbf{Unknown!}\\
\textbf{Description}: & PNG image data, 181 x 48, 8-bit/color RGBA, non-interlaced\\
\multicolumn{2}{l}{\ }\\
\textbf{Target}: & /usr/share/gdm/themes/Modern/centos-symbol.png\\
\textbf{Package}: & \textbf{Unknown!}\\
\textbf{Description}: & PNG image data, 48 x 48, 8-bit/color RGBA, non-interlaced\\
\multicolumn{2}{l}{\ }\\
\textbf{Target}: & /usr/share/gdm/themes/Modern/GdmGreeterTheme.desktop\\
\textbf{Package}: & \textbf{Unknown!}\\
\textbf{Description}: & UTF-8 Unicode English text\\
\multicolumn{2}{l}{\ }\\
\textbf{Target}: & /usr/share/gdm/themes/Modern/icon-language.png\\
\textbf{Package}: & \textbf{Unknown!}\\
\textbf{Description}: & PNG image data, 32 x 32, 8-bit/color RGBA, non-interlaced\\
\multicolumn{2}{l}{\ }\\
\textbf{Target}: & /usr/share/gdm/themes/Modern/icon-reboot.png\\
\textbf{Package}: & \textbf{Unknown!}\\
\textbf{Description}: & PNG image data, 32 x 32, 8-bit/color RGBA, non-interlaced\\
\multicolumn{2}{l}{\ }\\
\textbf{Target}: & /usr/share/gdm/themes/Modern/icon-session.png\\
\textbf{Package}: & \textbf{Unknown!}\\
\textbf{Description}: & PNG image data, 32 x 32, 8-bit/color RGBA, non-interlaced\\
\multicolumn{2}{l}{\ }\\
\textbf{Target}: & /usr/share/gdm/themes/Modern/icon-shutdown.png\\
\textbf{Package}: & \textbf{Unknown!}\\
\textbf{Description}: & PNG image data, 32 x 32, 8-bit/color RGBA, non-interlaced\\
\multicolumn{2}{l}{\ }\\
\textbf{Target}: & /usr/share/gdm/themes/Modern/Modern.xml\\
\textbf{Package}: & \textbf{Unknown!}\\
\textbf{Description}: & XML 1.0 document text\\
\multicolumn{2}{l}{\ }\\
\textbf{Target}: & /usr/share/gdm/themes/Modern/screenshot.png\\
\textbf{Package}: & \textbf{Unknown!}\\
\textbf{Description}: & PNG image data, 200 x 150, 8-bit/color RGBA, non-interlaced\\
\multicolumn{2}{l}{\ }\\
\hline
\end{longtable}


\section{Design}

The centos-release.png and screenshot.png images are rendered for each
major release of CentOS. This task is done using the rendering script
(render.sh) available in the workplace.  This script creates the
appropriate PNG images under img/\$VERSION/ directory. 

The background.png image is taken from Backgrounds section. This task
is done using the building script(build.sh) available in the
workplace. This script collects all information, groups it and stores
it under tgz/\$VERSION/\$RESOLUTION/ with the form \$THEME.tar.gz.

Whith the building script you can create GDM themes for specific
CentOS major releases, and inside each major release for specific
screen resolutions.

More information about GDM theming is available in the \emph{GNOME
Display Manager Reference Manual}. This guide is available online and
inside your system's help. As shortcut to get that help, you can run
the following command:

\begin{itemize}
\item gnome-help file:///usr/share/gnome/help/gdm/C/gdm.xml\#thememanual
\end{itemize}

\section{Configuration}

\begin{description}

\item[GraphicalTheme]: The graphical theme that the Themed Greeter
should use.  It should refer to a directory in the theme directory set
by \emph{GraphicalThemeDir}.

\texttt{GraphicalTheme=\$THEME}\\
                                              
\item[GraphicalThemeDir]: The directory where themes for the Themed
Greeter are installed.

\texttt{GraphicalThemeDir=/usr/share/gdm/themes/}

\item[BackgroundColor]: The Standard greeter (gdmlogin) background
color. If the BackgroundType is 2, use this color in the background of
the greeter.  Also use it as the back of transparent images set on the
background and if the BackgroundRemoteOnlyColor is set and this is a
remote display.  This only affects the GTK+ Greeter.

\texttt{BackgroundColor=\#204C8D}

\item[GraphicalThemeColor]: Use this color in the background of the
Themed Greeter. This only affects the Themed Greeter.

\texttt{GraphicalThemeColor=\#000000}

\end{description}

More information about GDM and its configuration can be found in its
reference guide. As shortcut to get that help, you can run the
following command:

\begin{itemize}
\item gnome-help file:///usr/share/gnome/help/gdm/C/gdm.xml\#index
\end{itemize}

\section{Rendering}
\section{Testing}
\section{Issues}

\begin{description}

\item[Transition from Greeter to Desktop]: This seems to be available
when using the variable \emph{BackgroundImage} in GTK+ Greeter only.
In Themed Greeter the best we have is the variable
\emph{GraphicalThemedColor} to specify the background color of the
transition.

\item[Different resolutions]: Designing of GDM theme needs to be
expandable through different screen resolutions. By default, GDM theme
uses a background image of 2048x1536 pixels. When screen resolution
changes the predifined behaviour is to scale this image to fit the
current display resolution. If the screen resolution is higher, or
differs in ratio (for example when it a wide screens) the design of
GDM them could loose quality or look different from the original one.

As a workaround, if GDM theme looks narrow or deformed to you, use the
building script to create the GDM theme in your specific screen
resolution.

\item[GDM theme installation]: Use the login screen administrator
(gdmsetup). This action requires you to have \emph{root} privileges.

\item[Default Display Manager]: By default GDM is the first display
manager choice\footnote{See the file /etc/X11/prefdm.}. If you
changed this and want to go back then, run the following command (as
\emph{root}) and reboot:

\texttt{echo "DISPLAYMANAGER=GNOME" > /etc/sysconfig/desktop}

\end{description}

% License section
\input{../../../../../Licenses/GFDL.tex}

\end{document}
