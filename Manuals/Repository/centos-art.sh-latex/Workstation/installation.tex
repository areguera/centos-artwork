% Part   : Preparing Your Workstation
% Chapter: Installation
% ------------------------------------------------------------
% $Id: installation.tex 6191 2010-08-02 02:36:14Z al $
% ------------------------------------------------------------

This chapter describes tools you need to have installed in your CentOS
workstation before using CentOS Artwork Repository.

\section{Subversion}

Subversion is a version control system, which allows you to keep old
versions of files and directories (usually source code), keep a log of
who, when, and why changes occurred, etc., like CVS, RCS or
SCCS.\footnote{More documentation about Subversion and its tools,
including detailed usage explanations of the svn, svnadmin, svnserve
and svnlook programs, historical background, philosophical approaches
and reasonings, etc., can be found at
\url{http://svnbook.red-bean.com/.}} 

To install Subversion client tools in your workstation you can use the
following command:

\begin{quote}
yum install subversion
\end{quote}

\section{Inkscape}

Inkscape is a GUI editor for Scalable Vector Graphics (SVG) format
drawing files, with capabilities similar to Adobe Illustrator,
CorelDraw, Visio, etc. Inkscape features include versatile shapes,
bezier paths, freehand drawing, multiline text, text on path, alpha
blending, arbitrary affine transforms, gradient and pattern fills,
node editing, SVG-to-PNG export, grouping, layers, live clones, and
more.

Note that Inkscape is not inside CentOS Distribution, so you need to
configure a third party repository like RPMForge or EPEL to install
Inkscape.  Installation of a third party repositories inside CentOS
Distribution is described in the following URL:

\begin{quote}
\url{http://wiki.centos.org/AdditionalResources/Repositories}
\end{quote}

Once you have configured the third party repository you can install
Inkscape using the following command:

\begin{quote}
yum install inkscape
\end{quote}

\section{ImageMagick}

ImageMagick is a free software suite for the creation, modification
and display of bitmap images. It can read, convert and  write images
in a large variety of formats. Images can be cropped, colors can be
changed, various effects can be applied, images can  be rotated  and
combined,  and text, lines, polygons, ellipses and Bézier curves can
be added to images and stretched and rotated.

To install ImageMagick in your workstation you can run the following
command:

\begin{quote}
yum install ImageMagick
\end{quote}

\section{Netpbm}

Netpbm is a toolkit for manipulation of graphic images, including
conversion of images between a variety of different formats.  There
are over 300 separate tools in the package including converters for
about 100 graphics formats.

To install Netpbm in your workstation you can run the following
command:

\begin{quote}
yum install netpbm\{-progs\}
\end{quote}

\section{Syslinux}

Syslinux is a suite of bootloaders, currently supporting DOS FAT
filesystems, Linux ext2/ext3 filesystems (EXTLINUX), PXE network boots
(PXELINUX), or ISO 9660 CD-ROMs (ISOLINUX).  It also includes a tool,
MEMDISK, which loads legacy operating systems from these media.  The
package \texttt{syslinux} provides the programs \texttt{ppmtolss16}
and \texttt{lss16toppm} which are used to produce Anaconda Prompt
images. The \texttt{ppmtolss16} Perl program also includes the file
format specification.

To install Syslinux in your workstation you can run the following
command:

\begin{quote}
yum install syslinux
\end{quote}

\section{GNU Image Manipulation Program}

GNU Image Manipulation Program (GIMP) is used to manipulate images
inside CentOS Artwork Repository.  

To install GIMP in your workstation you can run the following command:

\begin{quote}
yum install gimp
\end{quote}

\section{GNU Core Utilities}

The GNU core utilities are a set of tools commonly used in shell
scripts.

To install the GNU core utilities in your workstation you can run the
following command:

\begin{quote}
yum install core-utils
\end{quote}

\section{\LaTeX}

\LaTeX\ is a document preparation system implemented as a macro
package for Donald E.  Knuth's \TeX\ typesetting program. The \LaTeX\
command typesets a file of text using the \TeX\ program and the LaTeX
Macro package for \TeX.  To be more specific, it processes an input
file containing the text of a document with interspersed commands that
describe how the text should be formatted. 

To install \LaTeX\ in your workstation you can run the following
command:

\begin{quote}
yum install tetex-\{latex,fonts,doc,xdiv,dvips\}
\end{quote}
