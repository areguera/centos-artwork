The Backgrounds directory is probably the Themes' core compoent.
Inside Backgrounds/ directory you produce background images that are
reused by almost all theme's art works (e.g., Distribution, Websites,
Promotion, etc.).  

The Backgrounds/ directory structure can contain directories to help
you organize the design process. 

    Img/: In this directory is where you store all background images
          (e.g., .png, .jpg, .xpm, etc.).  This directory is required
          by rendering scripts.

    Tpl/: In this directory is where you store all scalable vector
          graphics (e.g., .svg) files. This directory is required by
          rendering scripts.

    Xcf/: In this directory is where you store all Gimp's project
          files (e.g, .xcf). This directory is optional. If you can
          create a beautiful background images using scalable vector
          graphics only, then there is no need to use Gimp to produce
          background images. Of course, you can merge Gimp's power
          with Inkscape's power to produce images based on them.  In
          this last case you need this directory.

Inside Backgrounds/ you can create your vectorial designs using
Inkscape and your background images using Gimp. Later you can export
your background image as png and load it in your vectorial design
using Inkscape's import feautre.  Note that you may need to repeat
this technic for different screen resoluions. In that case you need to
create one file for each screen resolution and do the appropriate
linking inside .svg to .png files.  For example if you need to produce
background images in 800x600 you need to create the following file:

    xcf/800x600.xcf

to produce the background image:

    img/800x600-bg.png

which is loaded in: 

    svg/800x600.svg

to produce the final background image:

    img/800x600.png         

The img/800x600.png background image is produced automatically by
means of rendering scripts.

In other cases, like Anaconda's, it is possible that you need to make
some variations to one background image that don't want to appear on
regular background images of the same resolution. In this case you
need to create a new and specific background image for that art
component.  For example, if you need to produce the background image
used by Anconda (800x600) art works you create the file:

    xcf/800x600-anaconda.xcf

to produce the background image:

    img/800x600-anaconda-bg.png

which is loaded in: 

    svg/800x600-anaconda.svg

to produce the file:

    img/800x600-anaconda.png

The 800x600-anaconda.png file is used by all Anaconda art works
sharing a common 800x600 screen resolution (e.g., Header, Progress,
Splash, Firstboot, etc.). The Anaconda Prompt is indexed to 16 colors
and 640x480 pixels so you need to create a 640x480 background image
for it, and take the color limitation into account when designing it.

Background images without artistic motif are generally used as based
to build the Background images that do contain the theme's artistic
motif. 

Background images are linked (using Inkscape's \textit{import}
feature) inside almost all theme art works. This structure let you
make centralized changes on the visual identity and propagate them
quickly to other areas. 

In this structure you design background images for different screen
resolutions based on theme's artistic motif.

