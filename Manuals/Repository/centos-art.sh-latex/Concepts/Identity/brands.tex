% Part   : Concepts
% Chapter: Corporate Identity
% ------------------------------------------------------------
% $Id: brands.tex 6207 2010-08-05 13:11:13Z al $
% ------------------------------------------------------------

\section{The CentOS Brand}
\hypertarget{sec:Concepts:Identity:Brands}{}
\label{sec:Concepts:Identity:Brands}

\begin{description}
\item[framework:] trunk/Identity/Brands/
\end{description}

\noindent The CentOS brand is the name or trademark that conncects the
producer with their products. In this case, the producer is The CentOS
Project and the products are the CentOS distributions, the CentOS web
sites, the CentOS promotion, etc.

The CentOS Project uses the CentOS brand inside its GNU/Linux
enterprise distributions, web sites, and promotions to connect them
all visually and this way committing the monolithic visual structure
where one unique name and one unique visual style is used in all
visual manifestations.

% ------------------------------------------------------------
\section{The CentOS Logotype}
\hypertarget{sec:Concepts:Identity:Brands:Logotype}{}
\label{sec:Concepts:Identity:Brands:Logotype}

\begin{description}
\item[framework:] trunk/Identity/Brands/Type
\end{description}

\noindent The CentOS Logotype is represented by the word ``CentOS''
using \texttt{denmark.ttf} typography. See
\autoref{fig:Concepts:Identity:Brands:Logotype}.

\begin{figure}
\begin{center}
\fbox{\includegraphics[width=0.8\textwidth]{%
   /home/centos/artwork/trunk/Identity/Brands/Img/CentOS/Type/Build/a/801.pdf}}
\end{center}
\caption{The CentOS Logotype.%
    \label{fig:Concepts:Identity:Brands:Logotype}}
\end{figure}

% ------------------------------------------------------------
\section{The CentOS Symbol}
\hypertarget{sec:Concepts:Identity:Brands:Symbol}{}
\label{sec:Concepts:Identity:Brands:Symbol}

\begin{description}
\item[framework:] trunk/Identity/Brands/Symbol
\end{description}

\noindent The CentOS Symbol is the main visual representation of The
CentOS Project, and probably the most importat visual component inside
CentOS corporate identity. See
\autoref{fig:Concepts:Identity:Brands:Symbol}.  Due the CentOS symbol
is graphical element, without any kind of embedded typography, it
provides an efficient way of identification in a multi-language
environments.

\begin{figure}
\begin{center}
\fbox{\includegraphics[width=0.8\textwidth]{%
   /home/centos/artwork/trunk/Identity/Brands/Img/CentOS/Symbol/Build/5c-a/801.pdf}}
\end{center}
\caption{The CentOS Symbol.%
    \label{fig:Concepts:Identity:Brands:Symbol}}
\end{figure}

% ------------------------------------------------------------
\section{The Concept Behind CentOS Symbol}
\hypertarget{sec:Concepts:Identity:Brands:SymbolConcept}{}
\label{sec:Concepts:Identity:Brands:SymbolConcept}

At the moment of writting these lines, I haven't found any reference
about the author who worked out the CentOS symbol and the concept
behind its design.  That information would be useful as motivation
source.  The CentOS symbol is the visual representation of that the
CentOS community is working for, it would be very nice to have that
information available somewhere.  Until then, all we can do is giving
interpretations about it.

I will take the adventure of describing my personal interpretation
about the CentOS symbol design and the concept behind it.  This
interpretation is not definite, nor a final concept. Certainly, this
interpretation may have nothing in common with the one used by the
author of CentOS symbol. The ideas written in this section may change
in the future in the sake of reaching a better CentOS symbol
interpretation for the CentOS community to stand on.\footnote{This is
probably an interesting topic to debate at
``\href{mailto:centos-devel@centos.org}{centos-devel@centos.org}''
mailing list.}

The first thing, in order to interpret the CentOS symbol, is to know
which is ``\hyperlink{sec:Concepts:CentOS:Mission}{The CentOS Project
Mission}'' (\autoref{sec:Concepts:CentOS:Mission}) and feel a deep
compromise with it.  Later on, take a look to the CentOS symbol and
try to identify each component its design is based on. If you take a
careful look at \autoref{fig:Concepts:Identity:Brands:Symbol} you find
that the CentOS symbol is based on squares, arrows and different
colors.

The square is a geometrical figure that has four parallel sides of
equal dimensions. The equal dimensions brings the idea of justice
among all parts involved. That is, each part is in harmony one
another. This kind of harmony could be verified at simple sight, or
you can take a rule and messure each side to see that they have the
same dimensions.  As long as we can verify this harmony is true, it
starts to be a fact of reason that we can rely on. 

In a second state, the CentOS symbol is built of four identical
$90^{\circ}$ squares filled with unique colors. The squares provide
reason based pragmatic facts. The colors provide emotions. So, in this
design state we could say that different emotions are controlled by
the same pragmatic reasons.

In a third state, the $90^{\circ}$ set of squares is duplicated to
create a new set of squares. In this new set of squares fill colors
were removed and the whole squares set was rotated $45^{\circ}$.  At
this point eight arrows, pointing the outside, are immediatly visible.
Emotions are so strong that they found a way to expand themselves out
of $90^{\circ}$ pragmatic reasons.  But reason evolves with changes
and takes new forms ---the $45^{\circ}$ squares set--- to let flow off
the emotions' nature, and thus, uses that enormous expansion force to
create an infinite loop of common benefits, still controlled by the
reason of pragmatic facts.

At this point the CentOS symbol has been completed.

% ------------------------------------------------------------
\section{The CentOS Trademark}
\hypertarget{sec:Concepts:Identity:Brands:Trademark}{}
\label{sec:Concepts:Identity:Brands:Trademark}

\begin{description}
\item[framework:] trunk/Identity/Brands/Type/Tpl/2c-tm.svg
\end{description}

\noindent The CentOS Trademark is a distinctive sign or indicator used
by The CentOS Project (as legal entity) to identify that its product
(The CentOS Distribution) or services to consumers with which the
trademark appears originate from a unique source, and to distinguish
its products or services from those of other entities.

\begin{figure}
\begin{center}
\fbox{\includegraphics[width=0.8\textwidth]{%
   /home/centos/artwork/trunk/Identity/Brands/Img/CentOS/Type/Build/tm/801.pdf}}
\end{center}
\caption{The CentOS Trademark.%
    \label{fig:Concepts:Identity:Brands:Trademark}}
\end{figure}

A trademark is designated by the following symbols:

\begin{itemize}

\item $^{\textup{\textsc{tm}}}$ (for an unregistered trademark, that
is, a mark used to promote or brand goods);

\item $^{\textup{\textsc{sm}}}$ (for an unregistered service mark,
that is, a mark used to promote or brand services); and
            
\item \textregistered\ (for a registered trademark).
            
\end{itemize}

% ------------------------------------------------------------
\section{The CentOS Release Trademark}
\hypertarget{sec:Concepts:Identity:Brands:Release}{}
\label{sec:Concepts:Identity:Brands:Release}

\begin{description}
\item[framework:] trunk/Identity/Brands/Type/Tpl/2c-tmr.svg
\end{description}

\noindent The CentOS Release Trademark combines the CentOS trademark
and one decimal number.  Based on
``\hyperlink{sec:Concepts:CentOS:Release}{The CentOS Release Schema}''
(\autoref{sec:Concepts:CentOS:Release}), the CentOS project uses the
CentOS release trademak to identify CentOS visual manifestations that
share common visual structures with internal differences (i.e., The
CentOS Distributions and their installation media).

Construction of CentOS release trademark, for major releases 4 and 5,
are illustrated on \autoref{fig:Concepts:Identity:Brands:Release:4}
and \autoref{fig:Concepts:Identity:Brands:Release:5}, respectively.

\begin{figure}
\begin{center}
\fbox{\includegraphics[width=0.8\textwidth]{%
   /home/centos/artwork/trunk/Identity/Brands/Img/CentOS/Type/Build/tmr4/801.pdf}}
\end{center}
\caption{The CentOS trademark for major release number four.%
    \label{fig:Concepts:Identity:Brands:Release:4}}
\end{figure}

\begin{figure}
\begin{center}
\fbox{\includegraphics[width=0.8\textwidth]{%
   /home/centos/artwork/trunk/Identity/Brands/Img/CentOS/Type/Build/tmr5/801.pdf}}
\end{center}
\caption{The CentOS trademark for major release number five.%
    \label{fig:Concepts:Identity:Brands:Release:5}}
\end{figure}

Another way is to copy the release trademark SVG artwork and paste it
on the SVG design template you want it to appear in. Done that,
replace the decimal number with the string \texttt{=MAJOR\_RELEASE=},
exactly.

When you render the artwork component, that where you pasted the
release trademark SVG artwork in, you are producing the same artwork
component design for as many major releases as you have specified in
the translation structure of that artwork component being rendered.
Note that, in order for this translation mechanism to work correctly,
the translation structure should be prepared to support the major
release schema first, as described in
``\hyperlink{cha:Concepts:Translations}{Translation}''
(\autoref{cha:Concepts:Translations}) and
``\hyperlink{sec:Concepts:CentOS:Release}{The CentOS Release Schema}''
(\autoref{sec:Concepts:CentOS:Release}).

% ------------------------------------------------------------
\section{The CentOS Logo}
\hypertarget{sec:Concepts:Identity:Brands:Logos}{}
\label{sec:Concepts:Identity:Brands:Logos}

\begin{description}
\item[framework:] trunk/Identity/Brands/Logos
\end{description}

\noindent The CentOS Logo is a graphical element (ideogram, symbol,
emblem, icon, sign) that, together with its logotype (a uniquely set
and arranged typeface) form The CentOS Trademark or commercial brand.
See \autoref{fig:Concepts:Identity:Brands:Logos:Horizontal}.

\begin{figure}
\begin{center}
\fbox{\includegraphics[width=0.8\textwidth]{%
   /home/centos/artwork/trunk/Identity/Brands/Img/CentOS/Logo/Horizontal/Build/5c-tm/801.pdf}}
\end{center}
\caption{The CentOS Logo (horizontal) with trademark (TM) included.%
    \label{fig:Concepts:Identity:Brands:Logos:Horizontal}}
\end{figure}
