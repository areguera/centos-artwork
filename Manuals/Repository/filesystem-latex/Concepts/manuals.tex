% Part   : Concepts
% Chapter: Manuals
% ------------------------------------------------------------
% $Id: manuals.tex 6191 2010-08-02 02:36:14Z al $
% ------------------------------------------------------------

\begin{description}
\item[framework:] trunk/Manuals/
\end{description}

\noindent This chapter describes the CentOS Artwork Repository User
Guide.  The CentOS Artwork Repository User Guide is the book you are
reading right now. The main goals of this book is helping you to
understand how CentOS Artwork Repository works, and what you can do to
get the best of it.  It is also an excuse for you to join us and help
improving it.

\section{Structure}

Inside CentOS Artwork Repository, documentation is conceived using
\LaTeX's book class. Instead of having the entire document in a single
file, information has been spread in separated files under Manuals
framework structure. The Manuals framework structure is illustrated in
\autoref{fig:Concepts:Manuals:Structure} and described in
``\hyperlink{sec:Concepts:Manuals:Files}{Relevant Files}'' (see
\autoref{sec:Concepts:Manuals:Files}) and
``\hyperlink{sec:Concepts:Manuals:Directories}{Relevant Directories}''
(see \autoref{sec:Concepts:Manuals:Directories}).

\begin{figure}[!hbp]
\hrulefill
\begin{verbatim}
trunk/Manuals/
|-- Concepts
|   |-- CentOS
|   |-- Frameworks
|   |-- Identity
|   `-- ...
|-- Distribution
|   |-- Anaconda
|   |   |-- Firstboot
|   |   |-- Header
|   |   |-- Progress
|   |   `-- ...
|   |-- Backgrounds
|   |-- BootUp
|   |   |-- GDM
|   |   |-- GRUB
|   |   `-- ...
|   `-- Release
|-- Licenses
|-- Translations
|-- Workstation
|-- convenctions.tex
|-- repository.aux
|-- repository.lof
|-- repository.log
|-- repository.lot
|-- repository.out
|-- repository.pdf
|-- repository.tex
`-- repository.toc
\end{verbatim}
\hrulefill
\caption{Manuals framework structure.%
   \label{fig:Concepts:Manuals:Structure}}
\end{figure}

\section{Writing Style}

When writing for CentOS Artwork Repository User Guide, keep in mind
the following quote taken from the online ``BBC News Styleguide'':
---The key to good writing is \textbf{simple thoughts simply
expressed}.  Use short sentences and short words.  Anything which is
confused, complicated, poorly written or capable of being
misunderstood risks losing the listener or viewer, and once you have
done that, you might just as well not have come to work---.

If you need to express complicated ideas, try to split them out in
smaller and simpler ideas as much as possible. If you consider it
appropriate, try to use
``\hyperlink{sec:Concepts:Identity:Models}{Design Models}'' (see
\autoref{sec:Concepts:Identity:Models}) to illustrate your thoughts.

\subsection{Cross References}

When you create \LaTeX's cross references, you need to define targets
and links.  Targets are the specific locations in the document that
links point to.  In \LaTeX, these cross reference targets and links
can be defined in many ways, so we need to standardize the way we use
inside CentOS Artwork Repository User Guide to make it look uniform
and easy to read.

Inside CentOS Artwork Repository User Guide, cross references look
like illustrated in
\autoref{fig:Concepts:Manuals:CrossRef:Presentation}.  Cross reference
targets are defined as illustrated in
\autoref{fig:Concepts:Manuals:CrossRef:Targets}, and links to those
targets are defined as illustrated in
\autoref{fig:Concepts:Manuals:CrossRef:Links}. 

Note that we use both \texttt{hypertarget} and \texttt{label} commands
to define targets, and \texttt{hyperlink} and \texttt{autoref} to
define links.  With \texttt{hyperlink} we create long text links
---usefull when reading in the coputer---, and with \texttt{autoref}
we create numbered links ---usefull when reading in a printed copy---.

\begin{figure}[!hbp]
\hrulefill
\begin{flushleft}
\dots you can find more information in
``\hyperlink{sec:Concepts:Identity:Brands}{Logos}'' (see
\autoref{sec:Concepts:Identity:Brands}), specifically in
\hyperlink{sec:Concepts:Identity:Brands:Logos}{the horizontal version} (see
\autoref{sec:Concepts:Identity:Brands:Logos}).
\end{flushleft}
\hrulefill
\caption{Cross reference link presentation.%
   \label{fig:Concepts:Manuals:CrossRef:Presentation}}
\end{figure}

\begin{figure}[!hbp]
\hrulefill
\begin{verbatim}
\part{Concepts}
...
\chapter{The CentOS Logo}
\hypertarget{sec:Concepts:Logo}{}
\label{sec:Concepts:Logo}
...
\section{Horizontal}
\hypertarget{sec:Concepts:Identity:Brands}{}
\label{sec:Concepts:Identity:Brands}
...
\end{verbatim}
\hrulefill
\caption{\LaTeX's definition for cross reference targets.%
   \label{fig:Concepts:Manuals:CrossRef:Targets}}
\end{figure}

\begin{figure}[!hbp]
\hrulefill
\begin{verbatim}
\dots you can find more information in
``\hyperlink{sec:Concepts:Identity:Brands}{The CentOS Logo}'' 
(see \autoref{sec:Concepts:Identity:Brands}), specifically in
\hyperlink{sec:Concepts:Identity:Brands:Logos}{the horizontal version} 
(see \autoref{sec:Concepts:Identity:Brands:Logos}).
\end{verbatim}
\hrulefill
\caption{\LaTeX's definition for cross reference links.%
   \label{fig:Concepts:Manuals:CrossRef:Links}}
\end{figure}

\subsection{Figures}

Inside CentOS Artwork Repository User Guide, illustrations (i.e.
images, framework structures, source code, commands, etc.) are shown
using \LaTeX's \texttt{figure} environment. An example of
\texttt{figure} environment definition is illustrated in
\autoref{fig:Concepts:Manuals:Figures}.  More information about
\LaTeX's \texttt{figure} environment can be found in \LaTeX's info
manual. To read the \LaTeX's info manual, execute in your terminal the
command: \texttt{info latex}.

\begin{figure}[!hbp]
\hrulefill
\begin{verbatim}
\begin{figure}[!hbp]
\hrulefill
...
\hrulefill
\caption{... .%
   \label{fig:...}}
\end{figure}
\end{verbatim}
\hrulefill
\caption{\LaTeX's definition for \texttt{figure} environment.%
   \label{fig:Concepts:Manuals:Figures}}
\end{figure}

\subsection{Tables}

Inside CentOS Artwork Repository User Guide, tabular information (i.e.
translation markers, etc.) is shown using \LaTeX's \texttt{table}
environment. An example of \texttt{table} environment definition is
illustrated in \autoref{fig:Concepts:Manuals:Tables}.  More
information about \LaTeX's \texttt{table} environment can be found in
\LaTeX's info manual. To read the \LaTeX's info manual, execute in
your terminal the command: \texttt{info latex}.

\begin{figure}[!hbp]
\hrulefill
\begin{verbatim}
\begin{table}[!hbp]
\centering
\begin{tabular}[pos]{cols}
\hline
...
\hline
\end{tabular}
\caption{... .%
   \label{tab:...}}
\end{table}
\end{verbatim}
\hrulefill
\caption{\LaTeX's definition for \texttt{table} environment.%
   \label{fig:Concepts:Manuals:Tables}}
\end{figure}

\section{Relevant Files}
\hypertarget{sec:Concepts:Manuals:Files}{}
\label{sec:Concepts:Manuals:Files}

\subsection{repository.tex}

The \texttt{repository.tex} file is the main book's file. Here is
where you define specific book information like class, title, authors,
etc.  Inside \texttt{repository.tex} you organize chapters and load
their sections.

\subsection{introduction.tex} 

The \texttt{Introduction.tex} file introduces a specific artwork
component: what it does, where and when it appears in, etc.

\subsection{framework.tex} 

The \texttt{rramework.tex} file describes how to interact with a
specific artwork component: where to find the artwork component inside
CentOS Artwork Repository, how to render their images, how to render
their translations, their specific translation markers, etc.

\subsection{rebranding.tex} 

The \texttt{rebranding.tex} file describes how to rebrand a specific
artwork component: where to find the arwork component inside CentOS
Distribution, related packages you need to modify, etc.

\section{Relevant Directories}
\hypertarget{sec:Concepts:Manuals:Directories}{}
\label{sec:Concepts:Manuals:Directories}

\subsection{Concepts}

The \texttt{Concepts} directory organizes chapters related to
``Concepts'' part.  Files in this directory describe concepts used
inside CentOS Artwork Repository.

\subsection{Workstation} 

The \texttt{Workstation} directory organizes chapters related to
``Preparing Your Workstation'' part. Files in this directory describe
actions (i.e. installation and configuration) you need to do before
using CentOS Artwork Repository. 

\subsection{Distribution} 

The \texttt{Distribution} directory organizes chapters releated to
``Distribution'' part. This part gets its attention into the different
artwork components of CentOS Distribution, using a subdirectory
structure to organize them and the files \texttt{introduction.tex},
\texttt{framework.tex}, and \texttt{rebranding.tex} to describe them.

\subsection{Licenses} 

The \texttt{Licenses} directory organizes licenses used in this book.

\section{Revisions}
\hypertarget{sec:Concepts:Manuals:Revisions}{}
\label{sec:Concepts:Manuals:Revisions}

Revisions are a way of organizing changes committed to CentOS Artwork
Repository User Guide. Revisions have the format ``Revision M.N'',
where M is the major revision number, and N is the update revision
number.  Revision update number (N) may increase by one every month to
release that month's changes.  Once the six month cycle is reached,
major revision number (M) is increased by one and update revision
number (N) is reset to 0.

\section{Export to PDF}

To produce the file \texttt{repository.pdf}, you need to get inside
the Manual's framework and execute the command:

\begin{quote}
\texttt{pdflatex repository.tex}
\end{quote}
