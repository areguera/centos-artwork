%    Part: Distribution
% Chapter: Anaconda Progress - Rebranding
% ------------------------------------------------------------
% $Id: rebranding.tex 6019 2010-06-26 06:42:08Z al $
% ------------------------------------------------------------

\section{Rebranding}

\subsection{Package: redhat-logos} 

The \texttt{redhat-logos} package contains files created by the CentOS
Project to replace the Red Hat ``Shadow Man'' logo and  RPM logo.  The
Red Hat ``Shadow Man'' logo, RPM, and the RPM logo are trademarks or
registered trademarks of Red Hat, Inc. Anaconda Progres images that
need to be rebranded in \texttt{redhat-logos} package are illustrated
in \autoref{fig:Distribution:Anaconda:Progress:Rebranding:Images}.

\begin{figure}[!hbp]
\hrulefill
\begin{verbatim}
/usr/share/anaconda/pixmaps/
|-- first-lowres.png
|-- first.png
|-- progress_first-lowres.png
|-- progress_first.png
|-- rnotes
|   |-- 01-centos5-welcome.png
|   |-- 02-centos5-donate.png
|   |-- 03-centos5-yum.png
|   |-- 04-centos5-repos.png
|   |-- 05-centos5-centosplus.png
|   |-- 06-centos5-support.png
|   |-- 07-centos5-docs.png
|   |-- 08-centos5-wiki.png
|   |-- 09-centos5-virtualization.png
|   |-- cs
|   |   |-- 01-centos5-welcome.png
|   |   |-- 02-centos5-donate.png
|   |   |-- 03-centos5-yum.png
|   |   |-- 04-centos5-repos.png
|   |   |-- 05-centos5-centosplus.png
|   |   |-- 06-centos5-support.png
|   |   |-- 07-centos5-docs.png
|   |   |-- 08-centos5-wiki.png
|   |   `-- 09-centos5-virtualization.png
|   |-- ... (more languages here)
\end{verbatim}
\hrulefill
\caption{Anaconda Progress slide images.%
   \label{fig:Distribution:Anaconda:Progress:Rebranding:Images}}
\end{figure}

Replacements for these files are available in the
\hyperlink{sec:Distribution:Anaconda:Progress:Identity:Image}{Anaconda
Progress image directory} (see
\autoref{sec:Distribution:Anaconda:Progress:Identity:Image}) of
\hyperlink{sec:Distribution:Anaconda:Progress:Identity}{Anaconda
Progress Identity} (see
\autoref{sec:Distribution:Anaconda:Progress:Identity}) inside your
working copy of CentOS Artwork Repository.

Once you rebrand the image files inside \texttt{redhat-logos} SRPM
package, you need to rebuild it with the new brand information.

\subsection{Package: centos-release-notes} 

During the installation process Anaconda provides a button labeled
``Release Notes'' (see
\autoref{fig:Distribution:Anaconda:Progress:Identity:Models:Slides}).
When this button is pressed the header and slide areas get hidden and
the available space is used to display CentOS release notes (see
\autoref{fig:Distribution:Anaconda:Progress:Identity:Models:Release}).

Presently, CentOS release notes are managed online and they don't
appear in Anaconda's release notes screen. A few paragraphs are used
instead to describe how CentOS release notes are managed and how they
can be accessed.

The \texttt{centos-release-notes} package contains Anaconda Progress
release notes files. Anaconda Progress release notes files are
illustrated in \autoref{fig:Distribution:Anaconda:Progress:Rebranding:ReleaseNotes:Files}.

\begin{figure}[!hbp]
\hrulefill
\begin{verbatim}
/usr/share/doc/centos-release-notes-5.2/
|-- RELEASE-NOTES-cs
|-- RELEASE-NOTES-cs.html
|-- RELEASE-NOTES-de
|-- RELEASE-NOTES-de.html
|-- RELEASE-NOTES-en
|-- RELEASE-NOTES-en.html
|-- RELEASE-NOTES-es
|-- RELEASE-NOTES-es.html
`-- ... (more language-specific release notes)
\end{verbatim}
\hrulefill
\caption{Anaconda Progress release notes files.%
   \label{fig:Distribution:Anaconda:Progress:Rebranding:ReleaseNotes:Files}}
\end{figure}

Files in
\autoref{fig:Distribution:Anaconda:Progress:Rebranding:ReleaseNotes:Files}
have their own framework inside CentOS Artwork Repository. Anaconda
Progress release notes are rendered similar to images, using templates
and translation files, as well as rendering scripts.  For more
information about release notes rendering see the chapter
``\hyperlink{cha:Distribution:ReleaseNotes}{Release Notes}''.
