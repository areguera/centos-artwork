% ------------------------------------------------------------
% $Id: release.tex 6024 2010-06-28 04:28:27Z al $
% ------------------------------------------------------------
    \section{The CentOS Release Schema}
\hypertarget{sec:Concepts:CentOS:Release}{}
      \label{sec:Concepts:CentOS:Release}

The upstream vendor has released 4 versions of enterprise Linux that
CentOS rebuilds the freely available SRPMS for. So, the major CentOS
releases are CentOS 2, CentOS 3, CentOS 4 and CentOS 5.  The upstream
vendor releases security updates as required by circumstances. CentOS
releases rebuilds of security updates as soon as possible. Usually
within 24 hours (our stated goal is with 72 hours, but we are usually
much faster).

The upstream vendor also releases numbered update sets for Version 3,
Version 4 and Version 5 of their product (Currently EL 3 update 9, EL
4 update 6 and EL 5 update 1) 2 to 4 times per year. There are new
ISOs from the upstream vendor provided for these update sets. Update
sets will be completed as soon as possible after the vendor releases
their version ... generally within 2 weeks. CentOS follows these
conventions as well, so CentOS 3.9 correlates with EL 3 update 9 and
CentOS 4.6 correlates with EL 4 update 6, CentOS 5.1 correlates to EL
5 update 1, etc.

One thing some people have problems understanding is that if you have
any CentOS-3 product and update it, you will be updated to the latest
CentOS-3.x version.

The same is true for CentOS-4 and CentOS 5. If you update any CentOS-4
product, you will be updated to the latest CentOS-4.x version, or to
the latest CentOS 5.x version if you are updating a CentOS 5 system.
This is exactly the same behavior as the upstream product. Let's
assume that the latest EL4 product is update 6. If you install the
upstream original EL4 CDs (the ones before any update set) and upgrade
via their up2date, you will have latest update set installed (EL4
update 6 in our example). Since all updates within a major release
(CentOS 2, CentOS 3, CentOS 4, CentOS 5) always upgrade to the latest
version when updates are performed (thus mimicking upstream behavior),
only the latest version is maintained in each main tree on the CentOS
mirrors.

There is a CentOS Vault containing old CentOS trees. This vault is a
picture of the older tree when it was removed from the main tree, and
does not receive updates. It should only be used for reference. 

